\documentclass[a4paper,fleqn]{article}

\usepackage[a4paper]{geometry}
\usepackage{NotesTeX}
%\geometry{verbose,tmargin=2.5cm,bmargin=2.5cm,lmargin=2.5cm,rmargin=2.5cm}

\setlength{\parskip}{\smallskipamount}
\setlength{\parindent}{0pt}

\usepackage{amsmath}
\usepackage{amssymb}
\usepackage{braket}

\usepackage[libertine]{newtxmath}
\usepackage[no-math]{fontspec}
\setmainfont{Linux Libertine O}
\setmonofont{JuliaMono-Regular}

\usepackage{hyperref}
\usepackage{url}
\usepackage{xcolor}

\usepackage[normalem]{ulem}

\usepackage{mhchem}

\usepackage{minted}
\newminted{julia}{breaklines,fontsize=\footnotesize}
\newminted{text}{breaklines,fontsize=\footnotesize}


\newcommand{\txtinline}[1]{\mintinline[fontsize=\footnotesize]{text}{#1}}
\newcommand{\jlinline}[1]{\mintinline[fontsize=\footnotesize]{julia}{#1}}

\definecolor{mintedbg}{rgb}{0.90,0.90,0.90}
\usepackage{mdframed}

%\BeforeBeginEnvironment{minted}{\begin{mdframed}[backgroundcolor=mintedbg,%
%  rightline=false,leftline=false,topline=false,bottomline=false]}
%\AfterEndEnvironment{minted}{\end{mdframed}}


\usepackage{tikz}
\usetikzlibrary{shapes.geometric}
\tikzstyle{mybox} = [draw=blue, fill=green!5, very thick, rectangle,
  rounded corners, inner sep=10pt, inner ysep=20pt]
\tikzstyle{fancytitle} = [fill=blue, text=white]

\renewcommand{\imath}{\mathrm{i}}

\begin{document}

\title{
  \begin{center}{
    \Huge \textit{PWDFT.jl}}\\
    {{\itshape Electronic Structure Calculations with Julia}}
\end{center}
}
\author{Fadjar Fathurrahman}


\affiliation{
Program Studi Teknik Fisika \\
Fakultas Teknologi Industri \\
Institut Teknologi Bandung
}

\emailAdd{fadjar.fathurrahman@gmail.com}
\emailAdd{fadjar@tf.itb.ac.id}

%\title{\textsf{PWDFT.jl} Documentation}
%\author{Fadjar Fathurrahman}
%\date{}

\maketitle

\newpage
\part{Introduction}

\txtinline{PWDFT.jl} is a package to solve electronic structure problems
based on density functional theory (DFT) and Kohn-Sham equations
It is written in Julia programming language.

The Kohn-Sham orbitals are expanded using plane wave basis. This basis set is
very popular within solid-state community and is also used in several electronic
structure package such as Quantum ESPRESSO, ABINIT, VASP, etc.

Features

- Total energy calculation of molecules, surfaces, and crystalline system
  within periodic unit cell (however, no corrections are 
  implemented for non-periodic systems yet).
- SCF with electron density mixing (for semiconducting and metallic systems)
- Direct minimization method using conjugate gradient (for semiconducting systems)
- GTH pseudopotentials (included in the repository)
- LDA-VWN and GGA-PBE functionals (via `Libxc.jl`)

\section{Installation}

Requirements

- [Julia: \url{https://julialang.org/downloads} version >= 0.7,
  with the following packages installed:
  - FFTW
  - SpecialFunctions
  - Libxc (a wrapper to Libxc \url{https://gitlab.com/libxc/libxc}
  - LibSymspg (a wrapper to spglib \url{https://github.com/atztogo/spglib}

These packages are registered so they can be installed by using Julia's package manager.

\begin{juliacode}
using Pkg
Pkg.add("FFTW")
Pkg.add("SpecialFunctions")
Pkg.add("Libxc")
Pkg.add("LibSymspg")
\end{juliacode}

These packages should be automatically installed `PWDFT.jl` is installed as
local package (see below).

Currently, this package is not yet registered. So, `Pkg.add("PWDFT")` will not work (yet).

We have several alternatives:

1. Using Julia's package manager to install directly from the repository URL:

```julia
Pkg.add(PackageSpec(url="https://github.com/f-fathurrahman/PWDFT.jl"))
```

2. Using Julia development directory. We will use \txtinline{$HOME/.julia/dev}
   for this.
   To enable \txtinline{$HOME/.julia/dev} directory, we need to modify the Julia's
  \txtinline{LOAD_PATH} variable. Add the following line in your
  \txtinline{$HOME/.julia/config/startup.jl}.

\begin{juliacode}
push!(LOAD_PATH, expanduser("~/.julia/dev"))
\end{juliacode}

  After this has been set, you can download the the package as zip file (using Github) or
  clone this repository to your computer.

  If you download the zip file, extract the zip file under
  \txtinline{$HOME/.julia/dev}. You need to rename the extracted directory
  to `PWDFT` (with no `.jl` extension).

Alternatively, create symlink under \txtinline{$HOME/.julia/dev}
to point to you cloned (or extracted) `PWDFT.jl` directory. The link name should not
contain the `.jl` part. For example:

\begin{textcode}
ln -fs /path/to/PWDFT.jl $HOME/.julia/dev/PWDFT    
\end{textcode}

3. Install PWDFT.jl as local package. Firstly, get into Pkg's REPL mode by tapping `]`,
and activate a independent environment `activate .` .

Install the PWDFT.jl package in this environment:

\begin{textcode}
(PWDFT) pkg> develop <path/to/PWDFT.jl>
\end{textcode}

To make sure that the package is installed correctly, you can load the package
and verify that there are no error messages during precompilation step.
You can do this by typing the following in the Julia console.

\begin{juliacode}
using PWDFT
\end{juliacode}


Change directory to \txtinline{examples/Si_fcc} and run the following in the terminal.

\begin{textcode}
julia run.jl
\end{textcode}

The above command will calculate total energy of hydrogen atom by SCF method.

The script will calculate total energy per unit cell of silicon crystal using
self-consistent field iteration and direct energy minimization.

Units

`PWDFT.jl` internally uses Hartree atomic units (energy in Hartree and length in bohr).

A simple work flow

- create an instance of `Atoms`:

\begin{juliacode}
atoms = Atoms(xyz_file="CH4.xyz", LatVecs=gen_lattice_sc(16.0))
\end{juliacode}

- create an instance of `Hamiltonian`:

\begin{juliacode}
ecutwfc = 15.0 # in Hartree
pspfiles = ["../pseudopotentials/pade_gth/C-q4.gth",
            "../pseudopotentials/pade_gth/H-q1.gth"]
Ham = Hamiltonian( atoms, pspfiles, ecutwfc )
\end{juliacode}

- solve the Kohn-Sham problem

\begin{juliacode}
KS_solve_SCF!( Ham, betamix=0.2 )  # using SCF (self-consistent field) method
# or
KS_solve_Emin_PCG!( Ham ) # direct minimization using preconditioned conjugate gradient
\end{juliacode}

More examples on creating an instance of \jlinline{Atoms}

GaAs crystal (primitive unit cell), using keyword \jlinline{xyz_string_frac}:
\begin{juliacode}
# Atoms
atoms = Atoms( xyz_string_frac=
    """
    2

    Ga  0.0   0.0   0.0
    As  0.25  0.25  0.25
    """,
    in_bohr=true,
    LatVecs = gen_lattice_fcc(10.6839444516)
)
\end{juliacode}


Hydrazine molecule in extended xyz file
\begin{juliacode}
atoms = Atoms(ext_xyz_file="N2H4.xyz")
\end{juliacode}
with the following `N2H4.xyz` file (generated using [ASE](https://wiki.fysik.dtu.dk/ase/)):
\begin{textcode}
6
Lattice="11.896428 0.0 0.0 0.0 12.185504 0.0 0.0 0.0 11.151965" Properties=species:S:1:pos:R:3:Z:I:1 pbc="T T T"
N       5.94821400       6.81171100       5.22639100        7 
N       5.94821400       5.37379300       5.22639100        7 
H       6.15929600       7.18550400       6.15196500        1 
H       5.00000000       7.09777800       5.00000000        1 
H       5.73713200       5.00000000       6.15196500        1 
H       6.89642800       5.08772600       5.00000000        1 
\end{textcode}

Lattice vectors information is taken from the xyz file.


More examples on creating an instance of `Hamiltonian`

Using 3x3x3 Monkhorst-Pack kpoint grid (usually used for crystalline systems):
\begin{juliacode}
Ham = Hamiltonian( atoms, pspfiles, ecutwfc, meshk=[3,3,3] )    
\end{juliacode}

Include 4 extra states:
\begin{juliacode}
Ham = Hamiltonian( atoms, pspfiles, ecutwfc, meshk=[3,3,3], extra_states=4 )
\end{juliacode}

Using spin-polarized (`Nspin=2 `):
\begin{juliacode}
Ham = Hamiltonian( atoms, pspfiles, ecutwfc, meshk=[3,3,3],
    Nspin=2, extra_states=4 )
\end{juliacode}

NOTES: Currently spin-polarized calculations are only supported by
specifying calculations with smearing scheme (no fixed magnetization yet),
so \jlinline{extra_states} should also be specified.


Using PBE exchange-correlation functional:
\begin{juliacode}
Ham = Hamiltonian( atoms, pspfiles, ecutwfc, meshk=[3,3,3],
    Nspin=2, extra_states=4, xcfunc="PBE" )
\end{juliacode}
Currently, only two XC functional is supported, namely `xcfunc="VWN"` (default) and
`xcfunc="PBE"`. Future developments should support all functionals included in LibXC.


More examples on solving the Kohn-Sham problem

Several solvers are available:

- \jlinline{KS_solve_SCF!}: SCF algorithm with density mixing

- \jlinline{KS_solve_SCF_potmix!}: SCF algorithm with XC and Hartree potential mixing

- \jlinline{KS_solve_Emin_PCG!}: using direct total energy minimization by preconditioned conjugate
  gradient method (proposed by Prof. Arias, et al.). Only
  the version which works with systems with band gap is implemented.

Stopping criteria is based on difference in total energy.


The following example will use \jlinline{Emin_PCG}.
It will stop if the difference in total energy is less than
\jlinline{etot_conv_thr} and it occurs twice in a row.
\begin{juliacode}
KS_solve_Emin_PCG!( Ham, etot_conv_thr=1e-6, NiterMax=150 )
\end{juliacode}

Using SCF with \jlinline{betamix} (mixing parameter) 0.1:
\begin{juliacode}
KS_solve_SCF!( Ham, betamix=0.1 )
\end{juliacode}

Smaller `betamix` usually will lead to slower convergence but more stable.
Larger `betamix` will give faster convergence but might result in unstable
SCF.

Several mixing methods are available in \jlinline{KS_solve_SCF!}:

- `simple` or linear mixing

- \jlinline{linear_adaptive}

- `anderson`

- `broyden`

- `pulay`

- `ppulay`: periodic Pulay mixing

- `rpulay`: restarted Pulay mixing


For metallic system, we use Fermi smearing scheme for occupation numbers of electrons.
This is activated by setting \jlinline{use_smearing=true} and specifying a small smearing parameter `kT`
(in Hartree, default `kT=0.001`).

\begin{juliacode}
KS_solve_SCF!( Ham, mix_method="rpulay", use_smearing=true, kT=0.001 )
\end{juliacode}


%Band structure calculations
%Please see
%[this](examples/bands_Si_fcc/run_bands.jl) as
%an example of how this can be obtained.

Citation

- Fadjar Fathurrahman, Mohammad Kemal Agusta, Adhitya Gandaryus Saputro, Hermawan Kresno Dipojono
  PWDFT.jl : A Julia package for electronic structure calculation using density functional theory and plane wave basis](https://doi.org/10.1016/j.cpc.2020.107372).
  Comp. Phys. Comm. **256** 107372 (2020).



\textbf{This document is a work in progress}

In this part I will describe my design choices in implementing \textsf{PWDFT.jl}.
This design is by no means perfect
and it might change in the future to accomodate more complex use cases.

\section{Overview}

The design of \textsf{PWDFT.jl} is intended to be rather simple. One constraint
that is set to the code is that it should be possible to perform application
of Hamiltonian operator to wave function as simple as:
%
\begin{juliacode}
Hpsi = Ham*psi # or
Hpsi = op_H(Ham, psi)
\end{juliacode}
%
where \jlinline{psi} is, currently, of type \jlinline{Array{ComplexF64,2}}
\footnote{This function may be extended take other types other that plain Julia
array for more complex case.}.
%
This comes with an important consequences: all other pieces of information
about how this operation is done should be present in the type of \jlinline{Ham}.
\footnote{We will also see some quirks related to this design choice later,
such as applying Hamiltonian to several k-points or spin-polarized case}.

In \textsf{PWDFT.jl}, the type of \jlinline{Ham} is \jlinline{Hamiltonian}.
Several important fields of \jlinline{Hamiltonian} are instances of the following
types (please refer to the source code for more details about this):
\begin{itemize}
\item \jlinline{Atoms}: contains information about atomic structure: cell
vectors, atomic species and atomic coordinates.
\item \jlinline{PsPot_GTH}: contains information about atomic pseudopotentials.
\item \jlinline{Electrons}: contains information about electronic states.
\item \jlinline{PWGrid}: contains information about plane wave basis set.
\item \jlinline{Potentials}: contains information about local potentials such
as local pseudopotential, Hartree and exchange-correlation potential.
\item \jlinline{PsPotNL}: contains information about nonlocal pseudopotential
terms.
\item \jlinline{Energies}: contains information about components of Kohn-Sham
energy.
\item \jlinline{SymmetryInfo}: contains information about symmetry operations.
\end{itemize}

\newpage
\part{Implementation}

\section{Atomic structure}
%
The type \jlinline{Atoms} contains the following information:
%
\begin{itemize}
\item Number of atoms: \jlinline{Natoms::Int64}
\item Number of atomic species: \jlinline{Nspecies::Int64}
\item Atomic coordinates: \jlinline{positions::Array{Float64,2}}
\item Unit cell vectors (lattice vectors): \jlinline{LatVecs::Array{Float64,2}}
\end{itemize}
%
\jlinline{Atoms} also contains several other fields such as \jlinline{Zvals}
which will be set according to the pseudopotentials assigned to
the instance of \jlinline{Atoms}.
\sidenote{\footnotesize{Not all fields of \jlinline{Atoms} (or any custom types defined in \textsf{PWDFT.jl})
are listed. The most up to date definition can be consulted in the corresping source code.}}

\input{images/Atoms_struct}

\jlinline{LatVecs} is a $3\times3$ matrix. The vectors are stored column-wise which is
opposite to the PWSCF input convention.
Several convenience functions to generate lattice vectors for Bravais lattices
are provided in \textsf{PWDFT.jl}. These functions adopt PWSCF definition. Some examples
are listed below.
\begin{itemize}
\item \jlinline{gen_lattice_sc} or \jlinline{gen_lattice_cubic} for generating
simple cubic lattice vectors.
\item \jlinline{gen_lattice_fcc}: for fcc structure
\item \jlinline{gen_lattice_bcc}: for bcc structure
\item \jlinline{gen_lattice_hcp}: for hcp structure
\end{itemize}
Please see file \txtinline{gen_lattice.jl} for more information.


There are several ways to initialize an instance of \jlinline{Atoms}. The following
are typical cases.
%
\begin{itemize}
%
\item From xyz file. We need to supply the path to xyz file as string and
set the lattice vectors:
%
\begin{juliacode}
atoms = Atoms(xyz_file="file.xyz", LatVecs=gen_lattice_sc(16.0))
\end{juliacode}
%
\item For crystalline systems, using keyword argument \jlinline{xyz_string_frac}
is sometimes convenient:
%
\begin{juliacode}
atoms = Atoms(xyz_string_frac=
        """
        2

        Si  0.0  0.0  0.0
        Si  0.25  0.25  0.25
        """, in_bohr=true,
        LatVecs=gen_lattice_fcc(10.2631))
\end{juliacode}
%
\textbf{IMPORTANT} We need to be careful to also specify \jlinline{in_bohr} keyword to get
the correct coordinates in bohr (which is used internally in \jlinline{PWDFT.jl}).
%
\item From extended xyz file, the lattice vectors information is included
along with several others information, if any, however they are ignored):
%
\begin{juliacode}
atoms = Atoms(ext_xyz_file="file.xyz")
\end{juliacode}
%
\end{itemize}


\section{Plane wave basis set, real space grid, and k-points}

The type \jlinline{PWGrid} wraps various variables related to plane wave basis
set. This has two fields of type \jlinline{GVectors}
and \jlinline{GVectorsW} for storing information about $\mathbf{G}$-vectors
that are used in potential and wave functions, respectively.

\input{images/PWGrid_struct}

We can define grid points over unit cell as:
$$
\mathbf{r} = \frac{i}{N_{s1}}\mathbf{a}_{1} + \frac{j}{N_{s2}}\mathbf{a}_{2} +
\frac{k}{N_{s3}}\mathbf{a}_{3}
$$
where $i = 0,1,\ldots,N_{s1}-1$, $j = 0,1,\ldots,N_{s2}-1$, $k = 0,1,\ldots,N_{s3}-1$

\input{images/GVectors_struct}

\input{images/GVectorsW_struct}

The $\mathbf{G}$-vectors can be defined as:
\begin{equation}
\mathbf{G} = n_1 \mathbf{b}_1 + n_2 \mathbf{b}_2 + n_3 \mathbf{b}_3
\end{equation}
where $n_1, n_2, n_3$ are integer numbers and
$\mathbf{b}_1, \mathbf{b}_2, \mathbf{b}_3$ are three vectors describing
unit cell of reciprocal lattice or \textit{unit reciprocal lattice vectors}.
They satisfy the following relations:
\begin{align*}
\mathbf{a}_1 & = 2\pi\frac{\mathbf{a}_{2} \times \mathbf{a}_{3}}{\Omega} \\
\mathbf{a}_2 & = 2\pi\frac{\mathbf{a}_{3} \times \mathbf{a}_{1}}{\Omega} \\
\mathbf{a}_3 & = 2\pi\frac{\mathbf{a}_{1} \times \mathbf{a}_{2}}{\Omega} \\
\end{align*}

A periodic function
\begin{equation}
f(\mathbf{r}) = f(\mathbf{r}+\mathbf{L}),\,\,\,
\mathbf{L} = n_{1}a_{1} + n_{2}a_{2} + n_{3}a_{3}
\end{equation}
can be expanded using plane wave basis basis functions as:
\begin{equation}
f(\mathbf{r}) = \frac{1}{\sqrt{\Omega}}\sum_{\mathbf{G}}
C_{\mathbf{G}} \exp(\imath \mathbf{G} \cdot \mathbf{r})
\end{equation}
where $C_{\mathbf{G}}$ are expansion coefficients. This sum is usually truncated
at a certain maximum value of $\mathbf{G}$-vector, $\mathbf{G}_{\mathrm{max}}$.

Kohn-Sham wave function:
\begin{equation}
\psi_{i,\mathbf{k}}(\mathbf{r}) = u_{i,\mathbf{k}}(\mathbf{r}) \exp\left[ \imath \mathbf{k} \cdot \mathbf{r} \right]
\end{equation}
where $u_{i,\mathbf{k}}(\mathbf{r}) = u_{i,\mathbf{k}}(\mathbf{r}+\mathbf{L})$

Using plane wave expansion:
\begin{equation}
u_{i,\mathbf{k}}(\mathbf{r}) =
\frac{1}{\sqrt{\Omega}}\sum_{\mathbf{G}} C_{i,\mathbf{k},\mathbf{G}} \exp(\imath \mathbf{G} \cdot \mathbf{r}),
\end{equation}
%
we have:
\begin{equation}
\psi_{i,\mathbf{k}}(\mathbf{r}) =
\frac{1}{\sqrt{\Omega}}\sum_{\mathbf{G}} C_{i,\mathbf{G+\mathbf{k}}}
\exp\left[ \imath (\mathbf{G}+\mathbf{k}) \cdot \mathbf{r} \right]
\end{equation}

With this expression we can expand electronic density in plane wave basis:
\begin{align*}
\rho(\mathbf{r}) & = \sum_{i} \int f_{i,\mathbf{k}}
\psi^{*}_{i,\mathbf{k}}(\mathbf{r}) \psi_{i,\mathbf{k}}(\mathbf{r})
\,\mathrm{d}\mathbf{k} \\
%
& = \frac{1}{\Omega} \sum_{i} \int f_{i,\mathbf{k}}
\left(
\sum_{\mathbf{G}'} C_{i,\mathbf{G'+\mathbf{k}}}
\exp\left[ -\imath (\mathbf{G}'+\mathbf{k}) \cdot \mathbf{r} \right]
\right)
%
\left(
\sum_{\mathbf{G}} C_{i,\mathbf{G+\mathbf{k}}}
\exp\left[ \imath (\mathbf{G}+\mathbf{k}) \cdot \mathbf{r} \right]
\right)
\,\mathrm{d}\mathbf{k} \\
%
& = \frac{1}{\Omega} \sum_{i} \int f_{i,\mathbf{k}}
\sum_{\mathbf{G}} \sum_{\mathbf{G}'}
C_{i,\mathbf{G+\mathbf{k}}} C_{i,\mathbf{G'+\mathbf{k}}}
\exp\left[ \imath (\mathbf{G}-\mathbf{G}') \cdot \mathbf{r} \right]
\,\mathrm{d}\mathbf{k} \\
%
& = \frac{1}{\Omega} \sum_{\mathbf{G}''}
C_{\mathbf{G}''} \exp\left[ \imath \mathbf{G}'' \cdot \mathbf{r} \right]
\,\mathrm{d}\mathbf{k}
\end{align*}
The sum over $\mathbf{G}''$ extends twice the range over the range needed
by the wave function expansion.

For wave function expansion we use plane wave expansion over $\mathbf{G}$
vectors defined by:
\begin{equation}
\frac{1}{2} \left| \mathbf{G} + \mathbf{k} \right|^2 \leq E_{\mathrm{cut}}
\label{eq:ecutwfc_def}
\end{equation}
where $E_{\mathrm{cut}}$ is a given cutoff energy which corresponds
to \jlinline{ecutwfc} field of \jlinline{PWGrid}.
For electronic density (and potentials) we have:
\begin{equation}
\frac{1}{2} \mathbf{G}^2 \leq 4 E_{\mathrm{cut}}
\label{eq:ecutrho_def}
\end{equation}
The value of $4 E_{\mathrm{cut}}$ corresponds to \jlinline{ecutrho} field of
of \jlinline{PWGrid}.

In the implementation, we first generate a set of $\mathbf{G}$-vectors which satisfies
Equation \eqref{eq:ecutrho_def} and derives several subsets from it which
satisfy Equation \eqref{eq:ecutwfc_def} for a given $\mathbf{k}$-points.

An instance of \jlinline{PWGrid} can be initialized by using its constructor
which has the following signature:
\begin{juliacode}
function PWGrid( ecutwfc::Float64, LatVecs::Array{Float64,2};
    kpoints=nothing, Ns_=(0,0,0) )
\end{juliacode}
There are two mandatory arguments: \jlinline{ecutwfc} and \jlinline{LatVecs}.
\jlinline{ecutwf} is cutoff energy for kinetic energy (in Hartree) and
\jlinline{LatVecs} is usually correspond to the one used in an
instance of \jlinline{Atoms}.

Structure factor for atomic species $I_{s}$ is calculated as
\begin{equation}
S_{f}(\mathbf{G},I_{s}) = \sum_{I} \exp\left[\mathbf{G} \cdot \mathbf{R}_{I_s}\right]
\end{equation}
where the summation is done over all atoms of species $I_s$.

Fast Fourier transforms are used to change the representation of a quantity from
real space to reciprocal space and \textit{vice versa}. They are:
\begin{itemize}
\item \jlinline{R_to_G}
\item \jlinline{G_to_R}
\end{itemize}
and also their inplace counterparts (\jlinline{R_to_G!} and \jlinline{G_to_R!}).

operators op nabla op nabla 2


The \jlinline{KPoints} struct stores variables related to $\mathbf{k}$-points
list.
\input{images/KPoints_struct}
For a total energy calculation, the list of $\mathbf{k}$-points is
generated using Monkhorst-Pack scheme.


\input{Electrons}

\section{Potentials and energies}

Total energy per unit cell system
$E^{\mathrm{KS}}_{\mathrm{total}}$ can be written as
\begin{equation}
E^{\mathrm{KS}}_{\mathrm{total}} =
E_{\mathrm{kin}} + E_{\mathrm{ele-nuc}} +
E_{\mathrm{Ha}} + E_{\mathrm{xc}} + E_{\mathrm{nuc-nuc}}
\label{eq:E_KS_total}
\end{equation}

Kohn-Sham equations:
\begin{equation}
H_{\mathrm{KS}} \psi_{i\mathbf{k}}(\mathbf{r}) =
\epsilon_{i\mathbf{k}} \psi_{i\mathbf{k}}(\mathbf{r})
\end{equation}

\input{images/Potentials_struct}

\input{images/Energies_struct}

\subsection{Electron density}

Electron density $\rho(\mathbf{r})$ is calculated as:
\begin{equation}
\rho(\mathbf{r}) = \sum_{i=1}^{N_{\mathrm{occ}}} f_{i} \psi^{*}_{i}(\mathbf{r})
\psi_{i}(\mathbf{r})
\end{equation}

Function: \jlinline{calc_rhoe!} and \jlinline{calc_rhoe}.

\subsection{Kinetic energy}

Kinetic energy:
\begin{equation}
E_{\mathrm{kin}} = -\frac{1}{2} \sum_{\mathbf{k}} \sum_{i}
w_{\mathbf{k}} f_{i\mathbf{k}}
\int_{\Omega}
\psi^{*}_{i\mathbf{k}}(\mathbf{r})
\nabla^2
\psi_{i\mathbf{k}}(\mathbf{r})
\,
\mathrm{d}\mathbf{r}
\label{eq:Kin_energy}
\end{equation}

In reciprocal space:
\begin{equation}
E_{\mathrm{kin}} =
\frac{1}{2} \sum_{\mathbf{k}} \sum_{i=1}^{N_{\mathrm{occ}}}
w_{\mathbf{k}} f_{i\mathbf{k}}
\sum_{\mathbf{G}} \left| \mathbf{G} + \mathbf{k} \right|^2
\left|c_{i,\mathbf{G}+\mathbf{k}}\right|^2
\end{equation}


\subsection{Local and nonlocal pseudopotential energy}

The local pseudopotential contribution is
\begin{equation}
E^{\mathrm{PS}}_{\mathrm{loc}} =
\int_{\Omega} \rho(\mathbf{r})\,V^{\mathrm{PS}}_{\mathrm{loc}}(\mathbf{r})\,
\mathrm{d}\mathbf{r}
\end{equation}
%
and the non-local contribution is
\begin{equation}
E^{\mathrm{PS}}_{\mathrm{nloc}} = 
\sum_{\mathbf{k}}
\sum_{i}
w_{\mathbf{k}}
f_{i\mathbf{k}}
\int_{\Omega}\,
\psi^{*}_{i\mathbf{k}}(\mathbf{r})
\left[
\sum_{I}\sum_{l=0}^{l_{\mathrm{max}}}
V^{\mathrm{PS}}_{l}(\mathbf{r}-\mathbf{R}_{I},\mathbf{r}'-\mathbf{R}_{I})
\right]
\psi_{i\mathbf{k}}(\mathbf{r})
\,\mathrm{d}\mathbf{r}.
\end{equation}

\subsection{Hartree energy}

\begin{equation}
E_{\mathrm{Ha}} = \frac{1}{2}
\int_{\Omega}
V_{\mathrm{Ha}}(\mathbf{r})\rho(\mathbf{r})\,
\mathrm{d}\mathbf{r}
\end{equation}
where the Hartree potential $V_{\mathrm{Ha}}$ is defined as
\begin{equation}
V_{\mathrm{Ha}}(\mathbf{r}) =
\int_{\Omega}
\frac{\rho(\mathbf{r}')}{\mathbf{r} - \mathbf{r}'}
\,\mathrm{d}\mathbf{r}'
\end{equation}
Alternatively, $V_{\mathrm{Ha}}$ can be calculated as the solution of Poisson
equation
\begin{equation}
\nabla^2 V_{\mathrm{Ha}}(\mathbf{r}) =
-4\pi\rho(\mathbf{r})
\end{equation}

\subsection{XC energy and potential}

\textsf{PWDFT.jl} uses \textsf{Libxc.jl}\cite{Libxc.jl}, a Julia wrapper to
\textsf{Libxc}\cite{Marques2012,Lehtola2018}, to calculate exchange correlation
energy and potentials.

For LDA we have:
\begin{align}
E_{\mathrm{xc}}\left[\rho_{\sigma}\right] & = \int \epsilon^{\mathrm{HEG}}_{\mathrm{xc}}
\left[ \rho_{\sigma}(\mathbf{r}) \right]
\rho_{\text{tot}}(\mathbf{r})\, \mathrm{d}\mathbf{r} \\
& = \int \left\{
\epsilon^{\mathrm{HEG}}_{\mathrm{x}} \left[ \rho_{\sigma}(\mathbf{r}) \right] +
\epsilon^{\mathrm{HEG}}_{\mathrm{c}} \left[ \rho_{\sigma}(\mathbf{r}) \right]
\right\}
\rho(\mathbf{r})\, \mathrm{d}\mathbf{r}
\end{align}

\begin{equation}
\delta E_{\mathrm{xc}}\left[\rho_{\sigma}\right] =
\sum_{\sigma} \int
\left(
\epsilon^{\mathrm{HEG}}_{\mathrm{xc}} +
\rho_{\mathrm{tot}} \frac{\partial}{\partial \rho_{\sigma}} \epsilon^{\mathrm{HEG}}_{\mathrm{xc}}
\right)
\, \mathrm{d}\mathbf{r}\,\delta \rho_{\sigma}
\end{equation}


\subsubsection{Calculation of $E_{\mathrm{xc}}$ in \textsf{PWDFT.jl} using Libxc}

Note:

For VWN functional (should be applicable to other LDA functionals), we have the following
for non-spin-polarized case:
%
\begin{juliacode}
function calc_epsxc_VWN( Rhoe::Array{Float64,1} )
  Npoints = size(Rhoe)[1]
  Nspin = 1
  eps_x = zeros(Float64,Npoints)
  eps_c = zeros(Float64,Npoints)

  ptr = Libxc.xc_func_alloc()
  
  # exchange part
  Libxc.xc_func_init(ptr, Libxc.LDA_X, Nspin)
  Libxc.xc_lda_exc!(ptr, Npoints, Rhoe, eps_x)
  Libxc.xc_func_end(ptr)

  # correlation part
  Libxc.xc_func_init(ptr, Libxc.LDA_C_VWN, Nspin)
  Libxc.xc_lda_exc!(ptr, Npoints, Rhoe, eps_c)
  Libxc.xc_func_end(ptr)

  Libxc.xc_func_free(ptr)

  return eps_x + eps_c
end
\end{juliacode}


\begin{juliacode}
function calc_epsxc_VWN( Rhoe::Array{Float64,2} )
  Nspin = size(Rhoe)[2]
  Npoints = size(Rhoe)[1]
  if Nspin == 1
    return calc_epsxc_VWN( Rhoe[:,1] )
  end

  # Do the transpose manually
  Rhoe_tmp = zeros(2*Npoints)
  ipp = 0
  for ip = 1:2:2*Npoints
    ipp = ipp + 1
    Rhoe_tmp[ip] = Rhoe[ipp,1]
    Rhoe_tmp[ip+1] = Rhoe[ipp,2]
  end

  # ....
  # The rest of the code are similar to non-spin polarized case,
  # however now we use `Nspin=2` and pass `Rhoe_tmp` instead of `Rhoe`
end
\end{juliacode}

For PBE or any gradient-corrected functionals:
\begin{juliacode}
function calc_epsxc_PBE( pw::PWGrid, Rhoe::Array{Float64,1} )
  Npoints = size(Rhoe)[1]
  Nspin = 1

  # calculate gRhoe2
  gRhoe = op_nabla( pw, Rhoe )
  gRhoe2 = zeros( Float64, Npoints )
  for ip = 1:Npoints
    gRhoe2[ip] = dot( gRhoe[:,ip], gRhoe[:,ip] )
  end

  eps_x = zeros(Float64,Npoints)
  eps_c = zeros(Float64,Npoints)

  ptr = Libxc.xc_func_alloc()

  # exchange part
  Libxc.xc_func_init(ptr, Libxc.GGA_X_PBE, Nspin)
  Libxc.xc_gga_exc!(ptr, Npoints, Rhoe, gRhoe2, eps_x)
  Libxc.xc_func_end(ptr)

  # correlation part
  Libxc.xc_func_init(ptr, Libxc.GGA_C_PBE, Nspin)
  Libxc.xc_gga_exc!(ptr, Npoints, Rhoe, gRhoe2, eps_c)
  Libxc.xc_func_end(ptr)

  Libxc.xc_func_free(ptr)

  return eps_x + eps_c
end
\end{juliacode}

For PBE spin-polarized case

\begin{juliacode}
function calc_epsxc_PBE( pw::PWGrid, Rhoe::Array{Float64,2} )
  Nspin = size(Rhoe)[2]
  if Nspin == 1
    return calc_epsxc_PBE( pw, Rhoe[:,1] )
  end

  Npoints = size(Rhoe)[1]

  # calculate gRhoe2
  gRhoe_up = op_nabla( pw, Rhoe[:,1] )
  gRhoe_dn = op_nabla( pw, Rhoe[:,2] )
  gRhoe2 = zeros( Float64, 3*Npoints )
  ipp = 0
  for ip = 1:3:3*Npoints
    ipp = ipp + 1
    gRhoe2[ip]   = dot( gRhoe_up[:,ipp], gRhoe_up[:,ipp] )
    gRhoe2[ip+1] = dot( gRhoe_up[:,ipp], gRhoe_dn[:,ipp] )
    gRhoe2[ip+2] = dot( gRhoe_dn[:,ipp], gRhoe_dn[:,ipp] )
  end

  Rhoe_tmp = zeros(2*Npoints)
  ipp = 0
  for ip = 1:2:2*Npoints
    ipp = ipp + 1
    Rhoe_tmp[ip] = Rhoe[ipp,1]
    Rhoe_tmp[ip+1] = Rhoe[ipp,2]
  end

  # ....
  # The rest of the code are similar to non-spin polarized case,
  # however now we use `Nspin=2` and pass `Rhoe_tmp` instead of `Rhoe`
end
\end{juliacode}


\subsubsection{Calculation of $V_{\mathrm{xc}}$ in \textsf{PWDFT.jl} using Libxc}

VWN non-spin polarized:
\begin{juliacode}
function calc_Vxc_VWN( Rhoe::Array{Float64,1} )
  Npoints = size(Rhoe)[1]
  Nspin = 1
  v_x = zeros(Float64,Npoints)
  v_c = zeros(Float64,Npoints)

  ptr = Libxc.xc_func_alloc()

  # exchange part
  Libxc.xc_func_init(ptr, Libxc.LDA_X, Nspin)
  Libxc.xc_lda_vxc!(ptr, Npoints, Rhoe, v_x)
  Libxc.xc_func_end(ptr)

  # correlation part
  Libxc.xc_func_init(ptr, Libxc.LDA_C_VWN, Nspin)
  Libxc.xc_lda_vxc!(ptr, Npoints, Rhoe, v_c)
  Libxc.xc_func_end(ptr)

  Libxc.xc_func_free(ptr)
  
  return v_x + v_c
end
\end{juliacode}

VWN spin-polarized:
\begin{juliacode}
function calc_Vxc_VWN( Rhoe::Array{Float64,2} )
  Nspin = size(Rhoe)[2]
  if Nspin == 1
    return calc_Vxc_VWN( Rhoe[:,1] )
  end

  Npoints = size(Rhoe)[1]

  Vxc = zeros( Float64, Npoints, 2 )
  V_x = zeros( Float64, 2*Npoints )
  V_c = zeros( Float64, 2*Npoints )

  # This is the transposed version of Rhoe, use copy
  Rhoe_tmp = zeros(2*Npoints)
  ipp = 0
  for ip = 1:2:2*Npoints
    ipp = ipp + 1
    Rhoe_tmp[ip] = Rhoe[ipp,1]
    Rhoe_tmp[ip+1] = Rhoe[ipp,2]
  end

  ptr = Libxc.xc_func_alloc()

  # exchange part
  Libxc.xc_func_init(ptr, Libxc.LDA_X, Nspin)
  Libxc.xc_lda_vxc!(ptr, Npoints, Rhoe_tmp, V_x)
  Libxc.xc_func_end(ptr)

  # correlation part
  Libxc.xc_func_init(ptr, Libxc.LDA_C_VWN, Nspin)
  Libxc.xc_lda_vxc!(ptr, Npoints, Rhoe_tmp, V_c)
  Libxc.xc_func_end(ptr)

  Libxc.xc_func_free(ptr)

  ipp = 0
  for ip = 1:2:2*Npoints
    ipp = ipp + 1
    Vxc[ipp,1] = V_x[ip] + V_c[ip]
    Vxc[ipp,2] = V_x[ip+1] + V_c[ip+1]
  end
  return Vxc
end
\end{juliacode}

PBE non-spin-polarized:
\begin{juliacode}
function calc_Vxc_PBE( pw::PWGrid, Rhoe::Array{Float64,1} )
  Npoints = size(Rhoe)[1]
  Nspin = 1

  # calculate gRhoe2
  gRhoe = op_nabla( pw, Rhoe )
  gRhoe2 = zeros( Float64, Npoints )
  for ip = 1:Npoints
    gRhoe2[ip] = dot( gRhoe[:,ip], gRhoe[:,ip] )
  end

  V_x = zeros(Float64,Npoints)
  V_c = zeros(Float64,Npoints)
  V_xc = zeros(Float64,Npoints)

  Vg_x = zeros(Float64,Npoints)
  Vg_c = zeros(Float64,Npoints)
  Vg_xc = zeros(Float64,Npoints)

  ptr = Libxc.xc_func_alloc()

  # exchange part
  Libxc.xc_func_init(ptr, Libxc.GGA_X_PBE, Nspin)
  Libxc.xc_gga_vxc!(ptr, Npoints, Rhoe, gRhoe2, V_x, Vg_x)
  Libxc.xc_func_end(ptr)

  # correlation part
  Libxc.xc_func_init(ptr, Libxc.GGA_C_PBE, Nspin)
  Libxc.xc_gga_vxc!(ptr, Npoints, Rhoe, gRhoe2, V_c, Vg_c)
  Libxc.xc_func_end(ptr)

  V_xc = V_x + V_c
  Vg_xc = Vg_x + Vg_c

  # gradient correction
  h = zeros(Float64,3,Npoints)
  divh = zeros(Float64,Npoints)
  for ip = 1:Npoints
    h[1,ip] = Vg_xc[ip] * gRhoe[1,ip]
    h[2,ip] = Vg_xc[ip] * gRhoe[2,ip]
    h[3,ip] = Vg_xc[ip] * gRhoe[3,ip]
  end
  # div ( vgrho * gRhoe )
  divh = op_nabla_dot( pw, h )
  for ip = 1:Npoints
    V_xc[ip] = V_xc[ip] - 2.0*divh[ip]
  end
  
  return V_xc
end
\end{juliacode}

PBE spin-polarized:
\begin{juliacode}
function calc_Vxc_PBE( pw::PWGrid, Rhoe::Array{Float64,2} )
  Nspin = size(Rhoe)[2]
  if Nspin == 1
    return calc_Vxc_PBE( pw, Rhoe[:,1] )
  end

  Npoints = size(Rhoe)[1]

  # calculate gRhoe2
  gRhoe_up = op_nabla( pw, Rhoe[:,1] ) # gRhoe = ∇⋅Rhoe
  gRhoe_dn = op_nabla( pw, Rhoe[:,2] )
  gRhoe2 = zeros( Float64, 3*Npoints )
  ipp = 0
  for ip = 1:3:3*Npoints
    ipp = ipp + 1
    gRhoe2[ip]   = dot( gRhoe_up[:,ipp], gRhoe_up[:,ipp] )
    gRhoe2[ip+1] = dot( gRhoe_up[:,ipp], gRhoe_dn[:,ipp] )
    gRhoe2[ip+2] = dot( gRhoe_dn[:,ipp], gRhoe_dn[:,ipp] )
  end

  V_xc = zeros(Float64, Npoints, 2)
  V_x  = zeros(Float64, Npoints*2)
  V_c  = zeros(Float64, Npoints*2)

  Vg_xc = zeros(Float64, 3, Npoints)
  Vg_x  = zeros(Float64, 3*Npoints)
  Vg_c  = zeros(Float64, 3*Npoints)

  Rhoe_tmp = zeros(2*Npoints)
  ipp = 0
  for ip = 1:2:2*Npoints
    ipp = ipp + 1
    Rhoe_tmp[ip] = Rhoe[ipp,1]
    Rhoe_tmp[ip+1] = Rhoe[ipp,2]
  end

  ptr = Libxc.xc_func_alloc()

  # exchange part
  Libxc.xc_func_init(ptr, Libxc.GGA_X_PBE, Nspin)
  Libxc.xc_gga_vxc!(ptr, Npoints, Rhoe_tmp, gRhoe2, V_x, Vg_x)
  Libxc.xc_func_end(ptr)

  # correlation part
  Libxc.xc_func_init(ptr, Libxc.GGA_C_PBE, Nspin)
  Libxc.xc_gga_vxc!(ptr, Npoints, Rhoe_tmp, gRhoe2, V_c, Vg_c)
  Libxc.xc_func_end(ptr)

  ipp = 0
  for ip = 1:2:2*Npoints
    ipp = ipp + 1
    V_xc[ipp,1] = V_x[ip] + V_c[ip]
    V_xc[ipp,2] = V_x[ip+1] + V_c[ip+1]
  end

  Vg_xc = reshape(Vg_x + Vg_c, (3,Npoints))

  h = zeros(Float64,3,Npoints)
  divh = zeros(Float64,Npoints)

  # spin up
  for ip = 1:Npoints
    h[1,ip] = 2*Vg_xc[1,ip]*gRhoe_up[1,ip] + Vg_xc[2,ip]*gRhoe_dn[1,ip]
    h[2,ip] = 2*Vg_xc[1,ip]*gRhoe_up[2,ip] + Vg_xc[2,ip]*gRhoe_dn[2,ip]
    h[3,ip] = 2*Vg_xc[1,ip]*gRhoe_up[3,ip] + Vg_xc[2,ip]*gRhoe_dn[3,ip]
  end
  divh = op_nabla_dot( pw, h )
  for ip = 1:Npoints
    V_xc[ip,1] = V_xc[ip,1] - divh[ip]
  end

  # Spin down
  for ip = 1:Npoints
    h[1,ip] = 2*Vg_xc[3,ip]*gRhoe_dn[1,ip] + Vg_xc[2,ip]*gRhoe_up[1,ip]
    h[2,ip] = 2*Vg_xc[3,ip]*gRhoe_dn[2,ip] + Vg_xc[2,ip]*gRhoe_up[2,ip]
    h[3,ip] = 2*Vg_xc[3,ip]*gRhoe_dn[3,ip] + Vg_xc[2,ip]*gRhoe_up[3,ip]
  end
  divh = op_nabla_dot( pw, h )
  for ip = 1:Npoints
    V_xc[ip,2] = V_xc[ip,2] - divh[ip]
  end
  
  return V_xc
end
\end{juliacode}


\section{Pseudopotentials}
%
Currently, \textsf{PWDFT.jl} supports a subset of GTH (Goedecker-Teter-Hutter)
pseudopotentials. This type of pseudopotential is analytic and thus is somewhat
easier to program.
%
\textsf{PWDFT.jl} distribution contains several parameters
of GTH pseudopotentials for LDA and GGA functionals.

These pseudopotentials can be written in terms of
local $V^{\mathrm{PS}}_{\mathrm{loc}}$ and
angular momentum $l$ dependent
nonlocal components $\Delta V^{\mathrm{PS}}_{l}$:
\begin{equation}
V_{\mathrm{ene-nuc}}(\mathbf{r}) =
\sum_{I} \left[
V^{\mathrm{PS}}_{\mathrm{loc}}(\mathbf{r}-\mathbf{R}_{I}) +
\sum_{l=0}^{l_{\mathrm{max}}}
V^{\mathrm{PS}}_{l}(\mathbf{r}-\mathrm{R}_{I},\mathbf{r}'-\mathbf{R}_{I})
\right]
\end{equation}

\input{images/PsPot_GTH_struct}

\subsection{Local pseudopotential}

The local pseudopotential for
$I$-th atom, $V^{\mathrm{PS}}_{\mathrm{loc}}(\mathbf{r}-\mathbf{R}_{I})$,
is radially symmetric
function with the following radial form
\begin{equation}
V^{\mathrm{PS}}_{\mathrm{loc}}(r) =
-\frac{Z_{\mathrm{val}}}{r}\mathrm{erf}\left[
\frac{\bar{r}}{\sqrt{2}} \right] +
\exp\left[-\frac{1}{2}\bar{r}^2\right]\left(
C_{1} + C_{2}\bar{r}^2 + C_{3}\bar{r}^4 + C_{4}\bar{r}^6
\right)
\label{eq:V_ps_loc_R}
\end{equation}
with $\bar{r}=r/r_{\mathrm{loc}}$ and $r_{\mathrm{loc}}$, $Z_{\mathrm{val}}$,
$C_{1}$, $C_{2}$, $C_{3}$ and $C_{4}$ are the corresponding pseudopotential
parameters.
In $\mathbf{G}$-space, the GTH local pseudopotential can be written as
\begin{multline}
V^{\mathrm{PS}}_{\mathrm{loc}}(G) = -\frac{4\pi}{\Omega}\frac{Z_{\mathrm{val}}}{G^2}
\exp\left[-\frac{x^2}{2}\right] +
\sqrt{8\pi^3} \frac{r^{3}_{\mathrm{loc}}}{\Omega}\exp\left[-\frac{x^2}{2}\right]\times\\
\left( C_{1} + C_{2}(3 - x^2) + C_{3}(15 - 10x^2 + x^4) + C_{4}(105 - 105x^2 + 21x^4 - x^6) \right)
\label{eq:V_ps_loc_G}
\end{multline}
where $x=G r_{\mathrm{loc}}$.

\subsection{Nonlocal pseudopotential}

\input{images/PsPotNL_struct}

The nonlocal component of GTH pseudopotential can written in real space as
\begin{equation}
V^{\mathrm{PS}}_{l}(\mathbf{r}-\mathbf{R}_{I},\mathbf{r}'-\mathbf{R}_{I}) =
\sum_{\mu=1}^{N_{l}} \sum_{\nu=1}^{N_{l}} \sum_{m=-l}^{l}
\beta_{\mu lm}(\mathbf{r}-\mathbf{R}_{I})\,
h^{l}_{\mu\nu}\,
\beta^{*}_{\nu lm}(\mathbf{r}'-\mathbf{R}_{I})
\end{equation}
where $\beta_{\mu lm}(\mathbf{r})$ are atomic-centered projector functions
\begin{equation}
\beta_{\mu lm}(\mathbf{r}) =
p^{l}_{\mu}(r) Y_{lm}(\hat{\mathbf{r}})
\label{eq:proj_NL_R}
\end{equation}
%
and $h^{l}_{\mu\nu}$ are the pseudopotential parameters and
$Y_{lm}$ are the spherical harmonics. Number of projectors per angular
momentum $N_{l}$ may take value up to 3 projectors.
%
In $\mathbf{G}$-space, the nonlocal part of GTH pseudopotential can be described by
the following equation.
\begin{equation}
V^{\mathrm{PS}}_{l}(\mathbf{G},\mathbf{G}') =
(-1)^{l} \sum_{\mu}^{3} \sum_{\nu}^{3}\sum_{m=-l}^{l}
\beta_{\mu l m}(\mathbf{G}) h^{l}_{\mu\nu}
\beta^{*}_{\nu l m}(\mathbf{G}')
\end{equation}
with the projector functions
\begin{equation}
\beta_{\mu lm}(\mathbf{G}) = p^{l}_{\mu}(G) Y_{lm}(\hat{\mathbf{G}})
\label{eq:betaNL_G}
\end{equation}
The radial part of projector functions take the following form
\begin{equation}
p^{l}_{\mu}(G) = q^{l}_{\mu}\left(Gr_{l}\right)
\frac{\pi^{5/4}G^{l}\sqrt{ r_{l}^{2l+3}}}{\sqrt{\Omega}}
\exp\left[-\frac{1}{2}G^{2}r^{2}_{l}\right]
\label{eq:proj_NL_G}
\end{equation}
%
For $l=0$, we consider up to $N_{l}=3$ projectors:
\begin{align}
q^{0}_{1}(x) & = 4\sqrt{2} \\
q^{0}_{2}(x) & = 8\sqrt{\frac{2}{15}}(3 - x^2) \\
q^{0}_{3}(x) & = \frac{16}{3}\sqrt{\frac{2}{105}} (15 - 20x^2 + 4x^4)
\end{align}
%
For $l=1$, we consider up to $N_{l}=3$ projectors:
\begin{align}
q^{1}_{1}(x) & = 8 \sqrt{\frac{1}{3}} \\
q^{1}_{2}(x) & = 16 \sqrt{\frac{1}{105}} (5 - x^2) \\
q^{1}_{3}(x) & = 8 \sqrt{\frac{1}{1155}} (35 - 28x^2 + 4x^4)
\end{align}
%
For $l=2$, we consider up to $N_{l}=2$ projectors:
\begin{align}
q^{2}_{1}(x) & = 8\sqrt{\frac{2}{15}} \\
q^{2}_{2}(x) & = \frac{16}{3} \sqrt{\frac{2}{105}}(7 - x^2)
\end{align}
%
For $l=3$, we only consider up to $N_{l}=1$ projector:
\begin{equation}
q^{3}_{1}(x) = 16\sqrt{\frac{1}{105}}
\end{equation}

In the present implementation, we construct the local and nonlocal
components of pseudopotential in the $\mathbf{G}$-space using
their Fourier-transformed expressions
and transformed them back to real space if needed.
We refer the readers to the original
reference \cite{Goedecker1996} and the book \cite{Marx2009}
for more information about GTH pseudopotentials.

Due to the separation of local and non-local components of electrons-nuclei
interaction, the electron-nuclei interaction energy can be written as
\begin{equation}
E_{\mathrm{ele-nuc}} = E^{\mathrm{PS}}_{\mathrm{loc}}
+ E^{\mathrm{PS}}_{\mathrm{nloc}}
\end{equation}
%
The local pseudopotential contribution is
\begin{equation}
E^{\mathrm{PS}}_{\mathrm{loc}} =
\int_{\Omega} \rho(\mathbf{r})\,V^{\mathrm{PS}}_{\mathrm{loc}}(\mathbf{r})\,
\mathrm{d}\mathbf{r}
\end{equation}
%
and the non-local contribution is
\begin{equation}
E^{\mathrm{PS}}_{\mathrm{nloc}} =
\sum_{\mathbf{k}}
\sum_{i}
w_{\mathbf{k}}
f_{i\mathbf{k}}
\int_{\Omega}\,
\psi^{*}_{i\mathbf{k}}(\mathbf{r})
\left[
\sum_{I}\sum_{l=0}^{l_{\mathrm{max}}}
V^{\mathrm{PS}}_{l}(\mathbf{r}-\mathbf{R}_{I},\mathbf{r}'-\mathbf{R}_{I})
\right]
\psi_{i\mathbf{k}}(\mathbf{r})
\,\mathrm{d}\mathbf{r}.
\end{equation}


\input{Hamiltonian}

\input{SCF}

\input{DirectMin}

\input{Forces}

\newpage
\part{Other things}

\appendix
\section{Howtos}

This part contains miscellaneous info.

TODO: Some of the should be moved into main text.

\input{HOWTOS.tex}


\section*{Status}

\textbf{29 July 2019} Total energy results are now similar to ABINIT
and Quantum ESPRESSO. A rather comprehensive test has been added
for SCF and Emin PCG for several simple systems.


\textbf{28 May 2018} The following features are working now:
\begin{itemize}
\item LDA and GGA, spin-paired and spin polarized calculations
\item Calculation with k-points (for periodic solids).
  \textsf{SPGLIB} is used to reduce the Monkhorst-Pack grid points
  for integration over Brillouin zone.
\end{itemize}

Band structure calculation is possible in principle as this can be
done by simply solving
Schrodinger equation with converged Kohn-Sham potentials, however there
is currently no tidy script or function to do that.

Total energy result for isolated systems (atoms and molecules) agrees quite
well with ABINIT and PWSCF results.

\sout{Total energy result for periodic solid is quite different from ABINIT and PWSCF.
I suspect that this is related to treatment of electrostatic terms in periodic system.}

These discrepancies have been minimized. For several systems the agreement is very good
even though I did not use the same algorithm as ABINIT.

\sout{SCF is rather shaky for several systems, however it is working in quite well in nonmetallic
system.}

SCF stability has been improved with Pulay mixing and its variants.



\bibliographystyle{unsrt}
\bibliography{BIBLIO}



\end{document}
