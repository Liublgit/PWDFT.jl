A simple work flow

- create an instance of `Atoms`:

\begin{juliacode}
atoms = Atoms(xyz_file="CH4.xyz", LatVecs=gen_lattice_sc(16.0))
\end{juliacode}

- create an instance of `Hamiltonian`:

\begin{juliacode}
ecutwfc = 15.0 # in Hartree
pspfiles = ["../pseudopotentials/pade_gth/C-q4.gth",
            "../pseudopotentials/pade_gth/H-q1.gth"]
Ham = Hamiltonian( atoms, pspfiles, ecutwfc )
\end{juliacode}

- solve the Kohn-Sham problem

\begin{juliacode}
KS_solve_SCF!( Ham, betamix=0.2 )  # using SCF (self-consistent field) method
# or
KS_solve_Emin_PCG!( Ham ) # direct minimization using preconditioned conjugate gradient
\end{juliacode}

More examples on creating an instance of \jlinline{Atoms}

GaAs crystal (primitive unit cell), using keyword \jlinline{xyz_string_frac}:
\begin{juliacode}
# Atoms
atoms = Atoms( xyz_string_frac=
    """
    2

    Ga  0.0   0.0   0.0
    As  0.25  0.25  0.25
    """,
    in_bohr=true,
    LatVecs = gen_lattice_fcc(10.6839444516)
)
\end{juliacode}


Hydrazine molecule in extended xyz file
\begin{juliacode}
atoms = Atoms(ext_xyz_file="N2H4.xyz")
\end{juliacode}

with the following \txtinline{N2H4.xyz} file (generated using Atomic Simulation
Environment) {\footnotesize \url{https://wiki.fysik.dtu.dk/ase/}}):

\begin{textcode}
6
Lattice="11.896428 0.0 0.0 0.0 12.185504 0.0 0.0 0.0 11.151965" Properties=species:S:1:pos:R:3:Z:I:1 pbc="T T T"
N       5.94821400       6.81171100       5.22639100        7 
N       5.94821400       5.37379300       5.22639100        7 
H       6.15929600       7.18550400       6.15196500        1 
H       5.00000000       7.09777800       5.00000000        1 
H       5.73713200       5.00000000       6.15196500        1 
H       6.89642800       5.08772600       5.00000000        1 
\end{textcode}

Lattice vectors information is taken from the xyz file.

More examples on creating an instance of \jlinline{Hamiltonian}

Using 3x3x3 Monkhorst-Pack kpoint grid (usually used for crystalline systems):
\begin{juliacode}
Ham = Hamiltonian( atoms, pspfiles, ecutwfc, meshk=[3,3,3] )    
\end{juliacode}

Include 4 extra states:
\begin{juliacode}
Ham = Hamiltonian( atoms, pspfiles, ecutwfc, meshk=[3,3,3], extra_states=4 )
\end{juliacode}

Using spin-polarized (\jlinline{Nspin=2}):
\begin{juliacode}
Ham = Hamiltonian( atoms, pspfiles, ecutwfc, meshk=[3,3,3],
    Nspin=2, extra_states=4 )
\end{juliacode}

NOTES: Currently spin-polarized calculations are only supported by
specifying calculations with smearing scheme (no fixed magnetization yet),
so \jlinline{extra_states} should also be specified.

Using PBE exchange-correlation functional:
\begin{juliacode}
Ham = Hamiltonian( atoms, pspfiles, ecutwfc, meshk=[3,3,3],
    Nspin=2, extra_states=4, xcfunc="PBE" )
\end{juliacode}
Currently, only two XC functional is supported, namely \jlinline{xcfunc="VWN"}
(default) and \jlinline{xcfunc="PBE"}.
Future developments should support all functionals included in LibXC.

More examples on solving the Kohn-Sham problem

Several solvers are available:
\begin{itemize}
\item \jlinline{KS_solve_SCF!}: SCF algorithm with density mixing
\item \jlinline{KS_solve_SCF_potmix!}: SCF algorithm with XC and Hartree potential mixing
\item \jlinline{KS_solve_Emin_PCG!}: using direct total energy minimization by
preconditioned conjugate gradient method (proposed by Prof. Arias, et al.). Only
the version which works with systems with band gap is implemented.
\end{itemize}

Stopping criteria is based on difference in total energy.

The following example will use \jlinline{Emin_PCG}.
It will stop if the difference in total energy is less than
\jlinline{etot_conv_thr} and it occurs twice in a row.
\begin{juliacode}
KS_solve_Emin_PCG!( Ham, etot_conv_thr=1e-6, NiterMax=150 )
\end{juliacode}

Using SCF with \jlinline{betamix} (mixing parameter) 0.1:
\begin{juliacode}
KS_solve_SCF!( Ham, betamix=0.1 )
\end{juliacode}

Smaller \jlinline{betamix} usually will lead to slower convergence but more stable.
Larger \jlinline{betamix} will give faster convergence but might result in unstable
SCF.

Several mixing methods are available in \jlinline{KS_solve_SCF!}:
\begin{itemize}
\item \jlinline{simple} or linear mixing
\item \jlinline{linear_adaptive}
\item \jlinline{anderson}
\item \jlinline{broyden}
\item \jlinline{pulay}
\item \jlinline{ppulay}: periodic Pulay mixing
\item \jlinline{rpulay}: restarted Pulay mixing
\end{itemize}

For metallic system, we use Fermi smearing scheme for occupation numbers of electrons.
This is activated by setting \jlinline{use_smearing=true} and
specifying a small smearing parameter \jlinline{kT}
(in Hartree, default \jlinline{kT=0.001}).

\begin{juliacode}
KS_solve_SCF!( Ham, mix_method="rpulay", use_smearing=true, kT=0.001 )
\end{juliacode}

%Band structure calculations
%Please see
%[this](examples/bands_Si_fcc/run_bands.jl) as
%an example of how this can be obtained.